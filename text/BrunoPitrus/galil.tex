% !TeX spellcheck = pl
% !TeX program = xelatex
\documentclass[a4paper]{article}
\usepackage[margin=12.7mm]{geometry}
\usepackage{enumitem,polyglossia,amsthm,mathtools,alltt}
\usepackage[math-style=ISO,colon=literal,mathrm=sym,mathit=sym,mathsf=sym,mathbf=sym,mathtt=sym]{unicode-math}
\setdefaultlanguage{polish}
\defaultfontfeatures[\rmfamily,\sffamily,\ttfamily]{}%turn off TeX ligatures
\setmainfont{Cambria}
\setmathfont{Cambria Math}
\setmonofont{Consolas}
\newtheorem*{theorem}{Twierdzenie}
\newtheorem*{lemma}{Lemat}
\setlength{\parindent}{0em}
\setlength{\parskip}{1em} %Akapity czasem są za krótkie, poniżej 1 linii więc tak będzie lepiej

\begin{document}
\title{Algorytm Galila–Seiferasa}
\author{Bruno Pitrus}
\date{}
\maketitle
Algorytm G–S jest algorytmem wyszukiwania wzorca który wymaga jedynie stałej pamięci dodatkowej (eliminując tablicę „mismatch” używaną przez algorytmy typu Morrisa–Pratta) a jednocześnie wciąż działa w asymptotycznym czasie liniowym — osiągając tym samym teoretyczne optimum pod względem zarówno czasu jaki i pamięci.

W tym opisie, tak samo jak w załączonej implementacji algorytmu, stringi indeksowane są od 1. Wszelki kod należy interpretować zgodnie z semantyką języka Python.

Niech \texttt{m = len(word)}, \texttt{n = len(text)}. Rozważmy schemat algorytmu dopasowania wzorca
\begin{alltt}
p,q= 0,0
\textbf{while} True:
  \textbf{while} text[p+q+1] == word[q+1]: q+= 1
  \textbf{if} q == m: \textbf{yield} p+1
  p,q= \textit{p'},\textit{q'}
\end{alltt}
W powyższym pseudokodzie \texttt{p} to jest obecny kandydat na pozycję wzorca, a \texttt{q} to prefiks o którym wiemy że się zgadza.

Naiwny algorytm \textit{brute-force} używa \texttt{p'= p+1}, \texttt{q'= 0}. Ponieważ wiemy jednak że \texttt{text[p+1:p+q+1] == word[1:q+1]} to rozważanie \texttt{p'= p+x} ma sens tylko jeśli \texttt{word[1:q-x+1] == word[x+1:q+1]}; na tej obserwacji jest oparty algorytm Knutha–Morrisa–Pratta, obliczający \texttt{p'= p+shift[q], q'= q-shift[q] if q>0 else 0}, gdzie \texttt{shift[q]} jest długością najkrótszego okresu \texttt{word[1:q+1]}.

Algorytm Galila–Seiferasa jest oparty na tym schemacie oraz następującym twierdzeniu:
\begin{theorem}[O dekompozycji]
Dla ustalonego (odpowiednio dużego) \(k\), każde słowo \texttt{x} można przedstawić jako \texttt{u+v}, gdzie \texttt{v} ma co najwyżej jeden \textit{\(k\)-okres prefiksu} a \texttt{u} jest długości \(\symrm O{\left(\text{długość najkrótszego okresu \texttt{v}}\right)}\). 
\end{theorem}
Słowo \texttt{z} nazywamy \(k\)-okresem prefiksu słowa \texttt{w} jeśli \texttt{z} jest słowem pierwotnym a \texttt{z\textsuperscript{k}} jest prefiksem \texttt{w}.

Do skonstruowania tej dekompozycji przyda nam się następujący lemat:
\begin{lemma}
Dla słowa \textit{pierwotnego} \texttt{w} istnieje dekompozycja \texttt{w == w1 + w2} taka że dla dowolnego słowa \texttt{w'}, słowo \texttt{w2 + w\textsuperscript{k-1} + w'} nie ma \(k\)-okresu prefiksu krótszego niż \texttt{len(w)}.
\end{lemma}

Pierwsza część algorytmu jest poświęcona znalezieniu dekompozycji wzorca takiej jak w twierdzeniu o dekompozycji.
\begin{alltt}
s= 0
\textbf{while} word[s+1:] \textit{ma więcej niż jeden k-okres prefiksu}:
  p2= \textit{długość drugiego najkrótszego k-okresu prefiksu}
  \textit{Korzystając z lematu, znajdź}
   s' < s + p2 \textit{takie że} word[s'+1:] \textit{nie ma k-okresu prefiksu krótszego niż} p2
  s= s'
\end{alltt}
Niech \texttt{l(s)} będzie długością najkrótszego okresu \texttt{word[s+1:]} a \texttt{p1(s)} długością najkrótszego \(k\)-okresu prefiksu. Indukcyjnie można dowieść, że w każdej iteracji pętli \texttt{s < c * min(l(s), p1(s))} dla \texttt{c = (k-1)/(k-2)} — więc w szczególności na końcu \texttt{len(u) < c * l(s)}.

W szczególności istnieje następujący prosty algorytm znajdowania \texttt{s'}:
\begin{alltt}
s'= s
\textbf{while} word[s'+1:] \textit{ma k-okres prefiksu krótszy niż p2}:
  usuń najkrótszy okres prefiksu
\end{alltt}
Przeanalizujmy złożoność ostatecznego kodu programu:
\begin{alltt}
K= 4
s,p1,q1= 0,1,0
p2,q2= 0,0
mode=0
\textbf{while} True:
  \textbf{if} mode==1:
    \textit{#Znajdź najkrótszy k-okres prefiksu word[s+1:]}
    \textbf{while} s+p1+q1 < m \textbf{and} word[s+p1+q1+1] == word[s+q1+1]: q1+= 1
    \textbf{if} p1+q1 >= p1*K: p2, q2= q1,0; mode=2; \textbf{continue}
    \textbf{if} s+p1+q1 == m: \textbf{break}
    p1+= (1 \textbf{if} q1==0 \textbf{else} (q1+K-1)//K); q1= 0
  \textbf{elif} mode==2:
    \textit{#Znajdź drugi najkrótszy k-okres prefiksu word[s+1:]. Jeśli nie istnieje to przejdź do drugiej fazy algorytmu.}
    \textbf{while} s+p2+q2 < m \textbf{and} word[s+p2+q2+1] == word[s+q2+1] \textbf{and} p2+q2 < p2*K: q2+= 1
    \textbf{if} p2+q2 == p2*K: mode= 0; \textbf{continue};
    \textbf{if} s+p2+q2 == m: \textbf{break}
    \textbf{if} q2 == p1+q1:
      p2+= p1; q2-= p1;
    \textbf{else}:
      p2+= (1 \textbf{if} q2==0 \textbf{else} (q2+K-1)//K); q2= 0
  \textbf{else}:
    \textit{#Zinkrementuj s}
    \textbf{while} s+p1+q1 < m \textbf{and} word[s+p1+q1+1] == word[s+q1+1]: q1+= 1
    \textbf{while} p1+q1 >= p1*K: s+= p1; q1-= p1;
    p1+= (1 \textbf{if} q1==0 \textbf{else} (q1+K-1)//K); q1= 0
    \textbf{if} p1 >= p2: mode= 1
\end{alltt}
W celu przeanalizowania złożoności tej części algorytmu rozważmy wyrażenie
\[
2s+{\left(k+1\right)}p_1+q_1+{\left(k+1\right)}p_2+q_2
\]
Jego wartość jest zawsze \(\symrm O{\left(\texttt{len(word)}\right)}\) a każde przypisanie ją zwiększa.
\begin{alltt}
p2,q2= 0,0
\textbf{while} True:
  \textbf{while} p2+s+q2 < n \textbf{and} s+q2 < m \textbf{and} text[p2+s+q2+1] == word[s+q2+1]: q2+= 1
  \textbf{if} q2 == m-s \textbf{and} text[p2+1 : p2+s+1] == word[1 : s+1]:
    yield p2+1
  \textbf{if} q2 == p1+q1:
    p2+= p1; q2-= p1
  \textbf{else}:
    p2+= (1 \textbf{if} q2==0 \textbf{else} (q2+K-1)//K); q2= 0
  \textbf{if} p2+s > n: \textbf{return}
\end{alltt}
Przed wykonaniem drugiej części algorytmu jeśli \texttt{word[s+1:]} ma \(k\)-okres prefiksu, to jego długość wynosi \(p_1\).
Ponadto twierdzimy, że \(s<{\left(k-1\right)}p_1/{\left(k-2\right)}\) co nam zagwarantuje że bezpośrednie sprawdzanie równości podciągów zajmie łącznie \(\symrm O{\left(\texttt{len(text)}\right)}\) czasu.
\begin{thebibliography}
{}Zvi Galil, Joel Seiferas, \textit{Time-Space-Optimal String Matching}, \textsc{Journal of Computer and System Sciences} \textbf{26}, 280–294 (1983)
\end{thebibliography}
\end{document}